%%% Fiktivní kapitola s ukázkami sazby

\chapter{Nápověda k~sazbě}

\section{Úprava práce}

Vlastní text diplomové práce je uspořádaný hierarchicky do kapitol a podkapitol,
každá kapitola začíná na nové straně. Text je zarovnán do bloku. Nový odstavec
se obvykle odděluje malou vertikální mezerou a odsazením prvního řádku. Grafická
úprava má být v~celém textu jednotná.

Práce se tiskne na bílý papír formátu A4. Okraje musí ponechat dost místa na vazbu:
doporučen je horní, dolní a pravý okraj $25\,\rm mm$, levý okraj $40\,\rm mm$.
Číslují se všechny strany kromě obálky a informačních stran na začátku práce;
první číslovaná strana bývá obvykle ta s~obsahem.

Písmo se doporučuje dvanáctibodové ($12\,\rm pt$) se standardní vzdáleností mezi řádky
(pokud píšete ve Wordu nebo podobném programu, odpovídá tomu řádkování $1,5$; v~\TeX{}u
není potřeba nic přepínat). Pro běžný text používejte vzpřímené patkové písmo.
Text matematických vět se obvykle tiskne pro zdůraznění skloněným (slanted) písmem,
není-li k~dispozici, může být zastoupeno kurzívou.

Primárně je doporučován jednostranný tisk (příliš tenkou práci lze obtížně svázat).
Delší práce je lepší tisknout oboustranně a přizpůsobit tomu velikosti okrajů:
$40\,\rm mm$ má vždy \emph{vnitřní} okraj. Rub titulního listu zůstává nepotištěný.

Zkratky použité v textu musí být vysvětleny vždy u prvního výskytu zkratky (v~závorce nebo
v poznámce pod čarou, jde-li o složitější vysvětlení pojmu či zkratky). Pokud je zkratek
více, připojuje se seznam použitých zkratek, včetně jejich vysvětlení a/nebo odkazů
na definici.

Delší převzatý text jiného autora je nutné vymezit uvozovkami nebo jinak vyznačit a řádně
citovat.

\section{Jednoduché příklady}

Čísla v~českém textu obvykle sázíme v~matematickém režimu s~desetinnou čárkou:
%%% Bez \usepackage{icomma}:
% $\pi \doteq 3{,}141\,592\,653\,589$.
%%% S \usepackage{icomma}:
$\pi \doteq 3,141\,592\,653\,589$.
V~matematických textech se považuje za přípustné používat desetinnou tečku
(pro lepší odlišení od čárky v~roli oddělovače). Numerické výsledky se uvádějí
s~přiměřeným počtem desetinných míst.

Mezi číslo a jednotku patří úzká mezera: šířka stránky A4 činí $210\,\rm mm$, což si
pamatuje pouze $5\,\%$ autorů. Pokud ale údaj slouží jako přívlastek, mezeru vynecháváme:
$25\rm mm$ okraj, $95\%$ interval spolehlivosti.

Rozlišujeme různé druhy pomlček:
červeno-černý (krátká pomlčka),
strana 16--22 (střední),
$45-44$ (matematické minus),
a~toto je --- jak se asi dalo čekat --- vložená věta ohraničená dlouhými pomlčkami.

V~českém textu se používají \uv{české} uvozovky, nikoliv ``anglické''.

% V tomto odstavci se vlnka zviditelňuje
{
\def~{{\tt\char126}}
Na některých místech je potřeba zabránit lámání řádku (v~\TeX{}u značíme vlnovkou):
u~předložek (neslabičnych, nebo obecně jednopísmenných), vrchol~$v$, před $k$~kroky,
a~proto, \dots{} obecně kdekoliv, kde by při rozlomení čtenář \uv{ško\-brt\-nul}.
}

\section{Matematické vzorce a výrazy}

Proměnné sázíme kurzívou (to \TeX{} v~matematickém módu dělá sám, ale
nezapomínejte na to v~okolním textu a také si matematický mód zapněte).
Názvy funkcí sázíme vzpřímeně. Tedy například:
$\var(X) = \E X^2 - \bigl(\E X \bigr)^2$.

Zlomky uvnitř odstavce (třeba $\frac{5}{7}$ nebo $\frac{x+y}{2}$) mohou
být příliš stísněné, takže je lepší sázet jednoduché zlomky s~lomítkem:
$5/7$, $(x+y)/2$.

Nechť
\[   % LaTeXová náhrada klasického TeXového $$
\mathbb{X} = \begin{pmatrix}
      \T{\bm x_1} \\
      \vdots \\
      \T{\bm x_n}
      \end{pmatrix}.
\]
Povšimněme si tečky za~maticí. Byť je matematický text vysázen
ve~specifickém prostředí, stále je gramaticky součástí věty a~tudíž je
zapotřebí neopomenout patřičná interpunkční znaménka. Výrazy, na které
chceme později odkazovat, je vhodné očíslovat:
\begin{equation}\label{eq01:Xmat}
\mathbb{X} = \begin{pmatrix}
      \T{\bm x_1} \\
      \vdots \\
      \T{\bm x_n}
      \end{pmatrix}.
\end{equation}
Výraz \eqref{eq01:Xmat} definuje matici $\mathbb{X}$. Pro lepší čitelnost
a~přehlednost textu je vhodné číslovat pouze ty výrazy, na které se
autor někde v~další části textu odkazuje. To jest, nečíslujte
automaticky všechny výrazy vysázené některým z~matematických
prostředí.

Zarovnání vzorců do několika sloupečků:
\begin{alignat*}{3}
S(t) &= \pr(T > t),    &\qquad t&>0       &\qquad&\text{ (zprava spojitá),}\\
F(t) &= \pr(T \leq t), &\qquad t&>0       &\qquad&\text{ (zprava spojitá).}
\end{alignat*}

Dva vzorce se spojovníkem:
\begin{equation}\label{eq01:FS}
\left.
\begin{aligned}
S(t) &= \pr(T > t) \\[1ex]
F(t) &= \pr(T \leq t)
\end{aligned}
\;	% zde pomůže ručně vynechat trochu místa
\right\}
\quad t>0 \qquad \text{(zprava spojité).}
\end{equation}

Dva centrované nečíslované vzorce:
\begin{gather*}
\bm Y = \mathbb{X}\bm\beta + \bm\varepsilon, \\[1ex]
\mathbb{X} = \begin{pmatrix} 1 & \T{\bm x_1} \\ \vdots & \vdots \\ 1 &
  \T{\bm x_n} \end{pmatrix}.
\end{gather*}
Dva centrované číslované vzorce:
\begin{gather}
\bm Y = \mathbb{X}\bm\beta + \bm\varepsilon, \label{eq02:Y}\\[1ex]
\mathbb{X} = \begin{pmatrix} 1 & \T{\bm x_1} \label{eq03:X}\\ \vdots & \vdots \\ 1 &
  \T{\bm x_n} \end{pmatrix}.
\end{gather}

Definice rozdělená na dva případy:
\[
P_{r-j}=
\begin{cases}
0, & \text{je-li $r-j$ liché},\\
r!\,(-1)^{(r-j)/2}, & \text{je-li $r-j$ sudé}.
\end{cases}
\]
Všimněte si použití interpunkce v této konstrukci. Čárky a tečky se
dávají na místa, kam podle jazykových pravidel patří.

\begin{align}
x& = y_1-y_2+y_3-y_5+y_8-\dots = && \text{z \eqref{eq02:Y}} \nonumber\\
& = y'\circ y^* = && \text{podle \eqref{eq03:X}} \nonumber\\
& = y(0) y' && \text {z Axiomu 1.}
\end{align}


Dva zarovnané vzorce nečíslované:
\begin{align*}
L(\bm\theta) &= \prod_{i=1}^n f_i(y_i;\,\bm\theta), \\
\ell(\bm\theta) &= \log\bigl\{L(\bm\theta)\bigr\} =
\sum_{i=1}^n \log\bigl\{f_i(y_i;\,\bm\theta)\bigr\}.
\end{align*}
Dva zarovnané vzorce, první číslovaný:
\begin{align}
L(\bm\theta) &= \prod_{i=1}^n f_i(y_i;\,\bm\theta), \label{eq01:L} \\
\ell(\bm\theta) &= \log\bigl\{L(\bm\theta)\bigr\} =
\sum_{i=1}^n \log\bigl\{f_i(y_i;\,\bm\theta)\bigr\}. \nonumber
\end{align}

Vzorec na dva řádky, první řádek zarovnaný vlevo, druhý vpravo, nečíslovaný:
\begin{multline*}
\ell(\mu,\,\sigma^2) = \log\bigl\{L(\mu,\,\sigma^2)\bigr\} =
\sum_{i=1}^n \log\bigl\{f_i(y_i;\,\mu,\,\sigma^2)\bigr\}= \\
  = -\,\frac{n}{2}\,\log(2\pi\sigma^2) \,-\,
\frac{1}{2\sigma^2}\sum_{i=1}^n\,(y_i - \mu)^2.
\end{multline*}

Vzorec na dva řádky, zarovnaný na $=$, číslovaný uprostřed:
\begin{equation}\label{eq01:ell}
\begin{split}
\ell(\mu,\,\sigma^2) &= \log\bigl\{L(\mu,\,\sigma^2)\bigr\} =
\sum_{i=1}^n \log\bigl\{f(y_i;\,\mu,\,\sigma^2)\bigr\}= \\
& = -\,\frac{n}{2}\,\log(2\pi\sigma^2) \,-\,
\frac{1}{2\sigma^2}\sum_{i=1}^n\,(y_i - \mu)^2.
\end{split}
\end{equation}

\section{Definice, věty, důkazy, \dots}

Konstrukce typu definice, věta, důkaz, příklad, \dots je vhodné
odlišit od okolního textu a~případně též číslovat s~možností použití
křížových odkazů. Pro každý typ těchto konstrukcí je vhodné mít
v~souboru s~makry (\texttt{makra.tex}) nadefinované jedno prostředí,
které zajistí jak vizuální odlišení od okolního textu, tak
automatické číslování s~možností křížově odkazovat.

\begin{definice}\label{def01:1}
  Nechť náhodné veličiny $X_1,\dots,X_n$ jsou definovány na témž
  prav\-dě\-po\-dob\-nost\-ním prostoru $(\Omega,\,\mathcal{A},\,\pr)$. Pak
  vektor $\bm X = \T{(X_1,\dots,X_n)}$ nazveme \emph{náhodným
    vektorem}.
\end{definice}

\begin{definice}[náhodný vektor]\label{def01:2}
  Nechť náhodné veličiny $X_1,\dots,X_n$ jsou definovány na témž
  pravděpodobnostním prostoru $(\Omega,\,\mathcal{A},\,\pr)$. Pak
  vektor $\bm X = \T{(X_1,\dots,X_n)}$ nazveme \emph{náhodným
    vektorem}.
\end{definice}
Definice~\ref{def01:1} ukazuje použití prostředí pro sazbu definice
bez titulku, definice~\ref{def01:2} ukazuje použití prostředí pro
sazbu definice s~titulkem.

\begin{veta}\label{veta01:1}
  Náhodný vektor $\bm X$ je měřitelné zobrazení prostoru
  $(\Omega,\,\mathcal{A},\,\pr)$ do $(\R_n,\,\mathcal{B}_n)$.
\end{veta}

\begin{lemma}[\citealp{Andel07}, str. 29]\label{veta01:2}
  Náhodný vektor $\bm X$ je měřitelné zobrazení prostoru
  $(\Omega,\,\mathcal{A},\,\pr)$ do $(\R_n,\,\mathcal{B}_n)$.
\end{lemma}
\begin{dukaz}
  Jednotlivé kroky důkazu jsou podrobně popsány v~práci \citet[str.
  29]{Andel07}.
\end{dukaz}
Věta~\ref{veta01:1} ukazuje použití prostředí pro sazbu matematické
věty bez titulku, lemma~\ref{veta01:2} ukazuje použití prostředí pro
sazbu matematické věty s~titulkem. Lemmata byla zavedena v~hlavním
souboru tak, že sdílejí číslování s~větami.
