\chapter{Machine Learning Review}



\section{Definitions}
Russel and Norvig succinctly define an agent and differnt types of learning in their book "Artificial Intelligence: A Modern Approach" (\cite{russell2009artificial}), their definition is paraphrased here. They define an agent as something that acts and a rational agent as one that acts so as to achieve the best outcome. If there there is uncertainty, then the agent tires to achieve the best expected outcome. Any component of an agent can be improved by learning from data. The improvements and techniques used to make them depend on four major factors:

\begin{itemize}
  \item Which component is to be improved.
  \item What prior knowledge the agent already has.
  \item What representation is used for the data and the component.
  \item What feedback is available to learn from.  
\end{itemize}

Here we will mostly be focused on the final point, "What feedback is available to learn from" but also slightly on the second point because we will incorporate Bayesian learning. There are three main types of feedback that determine the three main types of learning which are unsupervised, reinforcement, and supervised learning. 

In unsupervised learning an agent learns patterns even though no feedback is provided. In reinforcement learning, the agent learns from a series of rewards or punishments. In supervised learning, an agent learns from input-output pairs, which can be discrete or continuous, to find a function that maps the pairs. 

The goal of supervised learning is given a training set of $N$ example input-output pairs:

\[(x_1, y_1), (x_2,y_2),... (x_N,y_N),\]

where each $y_j$ was generated by some unknown function $y=f(x)$, find a function $h$ that approximates the true function
(\cite{russell2009artificial}).

In reality, the lines separating the types of learning aren't so clear. Semi-supervised learning is also an important and widely used method. In semi-supervised learning we are given a few labeled examples that were labled by some orcale (labeler, data annotator, etc.) and we must then make the most of a large collection of unlabeled examples. But what can we do with all the unlabeled data? This is where active learning comes in.



\section{Active Learning}
Supervised learning models almost always get more accurate with more labeled data. Active learning is the process of deciding which data to sample for annotation (\cite{munro2021human}). In other words, the central component of an active learning algorithm is the selection strategy, or deciding which of the unlabeled data could be the most useful to the model if it was labeled. Active learning uses a selection strategy that augments the existing classifier, it is not itself a classifier but rather a tool paired with a classifier.

Many strategies for choosing the next points to label exist. Here we will breifly define three basic approaches: uncertainty, diversity, and random sampling to get an idea of sampling. As well as two more advanced sampling approaches: xPAL and EER. When sampling the unlabled data an ordered list is returned and the top candidate is the candidate that is expected to be most valuable for the model, but we are not strictly limited to taking just one candidate.

\subsection{Random Sampling}
Random sampling is rather self explanatory as we randomly select an unlabeled data point from the pool and have an orcale provide a label.

\subsection{Diversity Sampling}
Diversity sampling is the set of strategies for identifying unlabeled items that are underrepresented or unknown to the machine learning model in its current state. The items may have features that are unique or obscure in the training data, or they might represent data that are currently under-represented in the model. 

Either way this can result in poor or uneven performance when the model is applied or the data is changing over time. The goal of diversity sampling is to target new, unusual, or underrepresented items for annotation to give the algorithm a more complete picture of the problem space (\cite{munro2021human}). 

\subsection{Uncertainty Sampling}
Uncertainty sampling is the set of strategies for identifying unlabeled items that are near a decision boundary in your current machine learning model. If you have a binary classification task, these items will have close to a 50\% probability of belonging to either label; therefore, the model is called uncertain or confused. 

These items are most likely to be wrongly classified, so they are the most likely to result in a label that differs from the predicted label, moving the decision boundary after they have been added to the training data and the model has been retrained (\cite{munro2021human}).

\subsection{xPAL}
Extended probabalistic gain for active learning (xPAL) is a decision-theoretic selection strategy that directly optimizes the gain and misclassification error, and uses a Bayesian approach by introducing a conjugate prior distribution to determine the class posterior to deal with uncertainties. Although the data distribution can be estimated, there is still uncertainty about the true class posterior probabilities. 

These class posterior probabilities can be modeled as a random variable based on the current observations in the dataset. For this model, a Bayesian approach is used by incorporating a conjugate prior to the observations. This produces more robust usefulness estimates for the candidates \cite{kottke2021toward}.

\subsection{EER}
Monte Carlo estimation of error reduction (EER) estimates future error rate by log-loss, using the entropy of the posterior class distribution on a sample fo the unlablled examples, or by 0-1 loss, using the posterior probabilities of the most probable class for the sampled unlabelled examples. 
\cite{roy2001toward}

\subsection{Summary}
After we select and label the candidate data we retrain the model with the updated data and we continue this process as new data is obtained or until we are satisfied with the performance of the model


